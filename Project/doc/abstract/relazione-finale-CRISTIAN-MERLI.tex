%%%%%%%%%%%%%%%%%%%%%%%%%%%%%%%%%%%%%%
%         PACKAGES INCLUSION         %                                                                                      % PACKAGES
%%%%%%%%%%%%%%%%%%%%%%%%%%%%%%%%%%%%%%

\documentclass{article}                                                                                                     % Document specs
\usepackage[legalpaper, margin=2cm]{geometry}                                                                               % Document margin
\usepackage[utf8]{inputenc}                                                                                                 % Encoding specs
\usepackage{hyperref}                                                                                                       % Package import for external references
\usepackage{graphicx}                                                                                                       % Package import to manage images labels and captions
\usepackage{cite}                                                                                                           % Package import for bibliographies (citations)
%%\usepackage{natbib}                                                                                                         % Package import command (citep citations)
%%\usepackage{amsmath}                                                                                                        % Package import command for formulas (math)
%%\usepackage{amssymb}                                                                                                        % Package import command for formulas (symbols)

%%%%%%%%%%%%%%%%%%%%%%%%%%%%%%%%%%%%%%
%           PREAMBLE START           %                                                                                      % PREAMBLE
%%%%%%%%%%%%%%%%%%%%%%%%%%%%%%%%%%%%%%

\setlength{\parindent}{0em}                                                                                                 % Remove indentation at paragraph's start
\hypersetup{colorlinks=true, linkcolor=red, urlcolor=blue}                                                                  % External references definition: red links and blue URLs (\href call)

\title{Dijkstra's algorithm implementation \\                                                                               % Title definition (printed with \maketitle command)
\large C project - G3 n.3, numerical calc and programming [145725] AY 2020/2021}                                            % Subtitle definition (printed with \maketitle command)
\author{Cristian Merli}                                                                                                     % Authors definition (printed with \maketitle command)
\date{20/07/2021}                                                                                                           % Date definition (printed with \maketitle command)

%%%%%%%%%%%%%%%%%%%%%%%%%%%%%%%%%%%%%%
% END OF PREAMBLE and DOCUMENT START %                                                                                      % DOC-START
%%%%%%%%%%%%%%%%%%%%%%%%%%%%%%%%%%%%%%

\begin{document}                                                                                                            % Document-start

\maketitle                                                                                                                  % Plot previously defined title

\begin{abstract}                                                                                                            % Abstract creation
  \noindent \textit{C-code implementation of Dijkstra's algorithm, inside a dedicated library to manage graphs.             % Abstract text
  This library has also been extended so that a graph's structure could be allocated inside heap to test
  Dijkstra’s algorithm. With the aim of getting a more user-friendly output, gnuplot takes care of plotting graphics
  to show the structure of the graph and the elaborated shortest path.}
\end{abstract}                                                                                                              % Abstract end

\vspace{1cm}                                                                                                                % Vertical-space command

\section{Project request}                                                                                                   % Section creation: "Project request"
  Dijkstra. Write a software which reads a graph and given two nodes, calculates the minimum path with Dijkstra’s           % Section text
  algorithm.
\label{sec:project_request}                                                                                                 % "project_request" reference-label definition

\section{Introduction}                                                                                                      % Section creation: "Introduction"
  This document has the main purpose of giving an overview of the project, deeping into theoretical aspects of              % Section text
  Dijkstra's algorithm and how it has been implemented in C-code. While to have further details about technical
  aspects, there is the possibility to consult html documentation of the software (see 'Doxygen html documentation'
  section inside 'README.md' file).
\label{sec:introduction}                                                                                                    % "introduction" reference-label definition

\section{Dijkstra's algorithm}                                                                                              % Section creation: "Dijkstra's algorithm"
  Dijkstra's algorithm has been conceived in 1956, by a Dutch computer sientist called Edsger Wybe Dijkstra.                % Section text
  The algorithm is capable of finding shortest path between two nodes (source and destination), inside a graph data
  structure. It has numerous applications in different fields: from gps-navigation (A* search), electrical/pipelines
  grids design, social networks suggestions, to AI applications as 'best-first search' (uniform cost search). This
  algorithm can be implemented in many different ways, adopting various data-structures and it comes in countless
  variants. In presented c-code library it has been chosen to take advantage of arrays, as data storing-structures.
  While as far as the algorithm itself is concened, it has been implemented in a variant to produce a shortest-path
  tree from source node, to all other nodes inside nodes collection-vector. That allows to be able to re-calculate
  more min-cost paths towards other nodes, without the need of running the algorithm again, if the source node remains
  unvaried.
\label{sec:dijkstra_algorithm}                                                                                              % "dijkstra_algorithm" reference-label definition

\section{Graph data-structure}                                                                                              % Section creation: "Graph data-structure"
  As mentioned in the abstract, graph-library does not contain only Dijkstra’s algorithm, but also a set of functions       % Section text
  to allow graph data-structure management \textit{(for further technical details, see doxygen html documentation)}.
  As briefly touched upon in previous chapter \textbf{[}\ref{sec:dijkstra_algorithm}\textbf{]}, for the purpose of
  storing \textbf{arches} and \textbf{nodes}, dynamic-memory vectors have been chosen. The major benefit of that
  consists of being able to resize the graph during runtime, adding or removing elements in allocated memory 
  (\textbf{nodes and arches collection-vectors inside heap}). Regarding arches and nodes, they are structure mainly
  made up of pointers to memory cells of connected elements, in addition to element-name and eventual cost. Besides,
  nodes also have an additional pointer to an element of \textbf{dijkstra-dataset dynamic-memory vector}, which contains
  a set of informations on which Dijkstra's algorithm works on. This way once the algorithm has been executed, it is
  possible to recontruct the shortest path backwards (from destination to source node), then to be later turned into a
  forward-path.
\label{sec:graph_data_structure}                                                                                            % "graph_data_structure" reference-label definition

\section{Library testing}                                                                                                   % Section creation: "Library testing"
  It is possible to test graph-library and Dijkstra's algorithm, through the developed main code (graph-test). There are    % Section text
  two different testing options:  
  \begin{itemize}                                                                                                           % List start code
    \item \textbf{Prepared test:}                                                                                           % List elements
    calculte shortest path using Dijkstra's algorithm from pre-defined source to pre-defined destination nodes.
    \item \textbf{Personalized test:}
    calculte shortest path using Dijkstra's algorithm from specified source to specified destination nodes.
  \end{itemize}                                                                                                             % List end code
\label{sec:library_testing}                                                                                                 % "library_testing" reference-label definition

\section{Gnuplot}                                                                                                           % Section creation: "Gnuplot"
  In order to get a more user-friendly output, cdcdscds.               % Section text
\label{sec:gnuplot}                                                                                                         % "gnuplot" reference-label definition

%\href{https://www.flaticon.com/free-icon/neural_2103658?related_id=2103633&origin=search#}{Icon link}

\end{document}                                                                                                              % End document code

%%%%%%%%%%%%%%%%%%%%%%%%%%%%%%%%%%%%%%
%            DOCUMENT END            %                                                                                      % DOC-END
%%%%%%%%%%%%%%%%%%%%%%%%%%%%%%%%%%%%%%
